\documentclass[a4paper]{scrartcl}

\usepackage{amsmath}
\usepackage{amsfonts}
\usepackage{amssymb}

\usepackage{amsthm}
\usepackage{stmaryrd}
\usepackage[all]{xy}
\usepackage{color}
\definecolor{darkblue}{rgb}{0,0,.5}
\usepackage[pdftex, plainpages = false, pdfpagelabels, colorlinks=true,
breaklinks=true, linkcolor=darkblue, menucolor=darkblue, pagecolor=darkblue, urlcolor=darkblue, citecolor=darkblue, anchorcolor=darkblue 
]{hyperref}
\usepackage[pdftex]{graphicx,color}
\usepackage{framed}
\usepackage{listings}

%typography extras :-)
\usepackage{microtype}
%\usepackage{ellipsis}

\title{Computational Geometry -- Exercise 4}
\author{Robert K\"unnemann (2512815)}

\begin{document}

\maketitle

\section*{Homogeneous Linear Kernel}

Here, we used the gcd-implementation of CGAL, that we installed in the second task. See \ref{homo_kernel} for the code.

\section*{Extend Linear Kernel }
\dots

\section*{Smallest Triangle of Intersection Points}
\dots


\appendix
\section{Code Listings}
\subsection{Homogeneous Kernel}
\label{homo_kernel}
\lstset{language=C++,emphstyle=\color{red},columns=fullflexible, keepspaces=true, showstringspaces=false, numbers=left, frame=single, breaklines=true, basicstyle=\small\ttfamily}
\lstinputlisting{homogeneous\_kernel/Homogenous\_kernel\_2.h}

\end{document}
